\documentclass[%draft,
    10pt,
    headinclude, footexclude,
    % twoside, % this produces strange margins!
    openright, % for new chapters
    notitlepage,
    cleardoubleempty,
    headsepline,
    pointlessnumbers,
    bibtotoc, idxtotoc,
    ]{scrbook}

% iPad Air 2 resolution: 1536 x 2048
% 1.3333 ratio (4:3)
% LeeSeshia is 15,73 ? 20,96 cm, ratio 1.332
%\setlength{\paperwidth}{15.73cm} \setlength{\paperheight}{20.96cm}
\setlength{\paperwidth}{15.72cm} \setlength{\paperheight}{20.95cm}
%\typearea{calc} % without BCOR results to a DIV of 8 for 11pt
\typearea[1.1cm]{15} % used 18, but 15 gives more space for the binding, TODO: check if 1.1 cm is the correct number

\usepackage{scrlayer-scrpage}
\setkomafont{pagehead}{\scshape\small}
\setkomafont{pagenumber}{\scshape\small}


\newif\ifbook
%\booktrue % comment out for the print book version

\ifbook
\else
\ifoot{\hyperlink{contents}{{\textnormal{Contents}}}}
\cfoot{\hyperlink{index}{{\textnormal{Index}}}}
\fi



\usepackage{pslatex} % -- times instead of computer modern, especially for the plain article class
\usepackage[pdftex,
            pdfauthor={Martin et al.},
            pdftitle={Introduction to Chip Design\\Using Open-Source Tools},
            colorlinks=true,bookmarks=true]{hyperref}

\usepackage{booktabs}
\usepackage{graphicx}
\usepackage{xcolor}
\usepackage{multirow}
\usepackage{cite}
\usepackage{dirtree}
\usepackage{pdfpages}
\usepackage{makeidx}
%\usepackage{showidx} % for index debugging
% For alignment on tables
\usepackage{dcolumn}
\newcommand{\cc}[1]{\multicolumn{1}{c}{#1}}
\usepackage{tikz}
\usetikzlibrary{arrows,bending}
\usetikzlibrary{positioning}


% fatter TT font
\renewcommand*\ttdefault{txtt}
% another TT, suggested by Alex
% \usepackage{inconsolata}
% \usepackage[T1]{fontenc} % needed as well?

% Smaller verbatim text
\usepackage{etoolbox}
\makeatletter
\patchcmd{\@verbatim}
  {\verbatim@font}
  {\verbatim@font\small}
  {}{}
\makeatother

\usepackage[procnames]{listings}


\ifbook
\hypersetup{
  linkcolor  = black,
  citecolor  = black,
  urlcolor   = black,
  colorlinks = black,
  bookmarks=false,
}
\else
\hypersetup{
  linkcolor  = blue,
  citecolor  = blue,
  urlcolor   = blue,
  colorlinks = true,
  bookmarks=true,
  pdfpagemode=UseOutlines,
}
\fi


\makeindex

% not really used
\newenvironment{comment}
{ \vspace{-0.1in}
  \begin{quotation}
  \noindent
  \small \em
  \rule{\linewidth}{0.5pt}\\
}
{
  \\
  \rule{\linewidth}{0.5pt}
  \end{quotation}
}

\newcommand{\scale}{0.7}

%\input{shared/chisel.tex}

\newcommand{\code}[1]{{\lstinline[basicstyle=\small\ttfamily]{#1}}}
\newcommand{\codefoot}[1]{{\lstinline[basicstyle=\footnotesize\ttfamily]{#1}}}

\newcommand{\todo}[1]{{\emph{TODO: #1}}}
\newcommand{\martin}[1]{{\color{blue} Martin: #1}}
\newcommand{\myref}[2]{\href{#1}{#2}}

\ifbook
\renewcommand{\myref}[2]{{#2}{\footnote{\url{#1}}}}
\fi

% uncomment following for final submission
%\renewcommand{\todo}[1]{}
%\renewcommand{\martin}[1]{}

\makeindex

\begin{document}


\ifbook
\else
%\includepdf{chisel-cover.pdf}
\newpage
\thispagestyle{empty}
~
\newpage
\fi



\begin{flushleft}
\pagestyle{empty}
\ \\
\vspace{1cm}
{\usekomafont{title}\mdseries\huge Introduction to Chip Design\\ \large Using Open-Source Tools}
\ \\
\vspace{1cm}
{\usekomafont{title}\mdseries\Large First Edition}
\cleardoublepage
\end{flushleft}
\newpage


\begin{flushleft}
\pagestyle{empty}
\ \\
\vspace{1cm}
{\usekomafont{title}\Huge Introduction to Chip Design\\ \large Using Open-Source Tools\\
\bigskip
{\usekomafont{title}\huge First Edition}\\
\bigskip
\bigskip
\bigskip
\bigskip
%{\large\itshape Beta Edition}\\
\bigskip
{\usekomafont{title}\huge Martin et al.}
\medskip\\
%{\large\itshape martin@jopdesign.com}

}

%\vspace{10cm} \emph{Version: \today}
\newpage
\end{flushleft}

\thispagestyle{empty}
\begin{flushleft}
{\small

%\lowertitleback{
Copyright \copyright{} 2024-2025 Martin...
  \medskip\\
  \begin{tabular}{lp{.8\textwidth}}
    \raisebox{-12pt}{\includegraphics[height=18pt]{figures/cc_by_sa}} &
     This work is licensed under a Creative Commons Attribution-ShareAlike
     4.0 International License.
     \url{http://creativecommons.org/licenses/by-sa/4.0/}\\
  \end{tabular}
%}

\medskip

Email: \url{martin@jopdesign.com}\\
Visit the source at \url{https://github.com/os-chip-design/chip-design-book}
\medskip

First edition published 2025 by Kindle Direct Publishing,\\
\url{https://kdp.amazon.com/}
\medskip
\medskip


\textbf{Library of Congress Cataloging-in-Publication Data}
\medskip

TBD... Schoeberl, Martin
\begin{quote}
xxxl\\
Martin ...\\
Includes bibliographical references and an index.\\
ISBN xxx
\end{quote}

\bigskip


Manufactured in the United States of America.

Typeset by Martin Schoeberl.}
\end{flushleft}

\frontmatter

\phantomsection
\hypertarget{contents}{}
\tableofcontents


\begingroup
\let\cleardoublepage\clearpage
\listoffigures
\listoftables
\lstlistoflistings
\endgroup

\chapter{Foreword}

\medskip
\medskip

It is an exciting time to be in the world of digital design....

\chapter{Preface}

% This text goes on the backside of the book, and in Amazon description
This book is an introduction to chip design with a focus on using...





\section*{Acknowledgements}

\todo{Maybe? Probably not yet ;-)}


\mainmatter

\chapter{Introduction}
\label{sec:intro}

This book is an introduction to chip design using open-source tools.


\chapter{The MOSFET and CMOS}

\chapter{Building Standard Cells}

\section{FABs}

\section{PDK}

\chapter{The Design Flow}

\chapter{Hardware Description Languages}

\section{Verilog}
\section{VHDL}
\section{SystemVerilog}
\section{Chisel}

\cite{chisel:dac2012} \cite{chisel:book}
\section{Other Languages}
\subsection{Amaranth}
\subsection{SpinalHDL}
\subsection{MyHDL}
\subsection{Clash}
\section{Generator Scripting Languages}

\chapter{Open-Source Tools}


\chapter{Open-Source Production Frameoworks}

\section{Caravel and efabless}

\section{Tiny Tapeout}

\section{Maybe something from Edu4Chip?}

\chapter{xxx}


\chapter{Acronyms}

Hardware designers and computer engineers like to use acronyms.
However, it takes time to get used to them. Here is a list of common terms
related to digital design and computer architecture.

\todo{Fix for the very many acronyms we use in chip design...}


\begin{description}
\item [ADC] analog-to-digital converter
\item [ALU] arithmetic and logic unit
\item [ASIC] application-specific integrated circuit
\item [CFG] control flow graph
\item [Chisel] constructing hardware in a Scala embedded language
\item [CISC] complex instruction set computer
\item [CPI] clock cycles per instruction
\item [CPU] central processing unit
\item [CRC] cyclic redundancy check
\item [DAC] digital-to-analog converter
\item [DFF] D flip-flop, data flip-flop
\item [DMA] direct memory access
\item [DRAM] dynamic random access memory
\item [EMC] electromagnetic compatibility
\item [ESD] electrostatic discharge
\item [FF] flip-flop
\item [FIFO] first-in, first-out
\item [FPGA] field-programmable gate array
% \item [GC] garbage collect(ion/or)
\item [HDL] hardware description language
\item [HLS] high-level synthesis
\item [IC] instruction count
\item [IDE] integrated development environment
\item [ILP] instruction level parallelism
\item [IC] integrated circuit
\item [IO] input/output
\item [ISA] instruction set architecture
\item [JDK] Java development kit
\item [JIT] just-in-time
\item [JVM] Java virtual machine
\item [LC] logic cell
\item [LRU] least-recently used
\item [LSB] least significant bit
\item [MMIO] memory-mapped IO
\item [MSB] most significant bit
\item [MUX] multiplexer
\item [OO] object oriented
\item [OOO] out-of-order
\item [OS] operating system
\item [RAM] random access memory
\item [RISC] reduced instruction set computer
%\item [RT] Real-Time
%\item [RTOS] Real-Time Operating System
\item [SDRAM] synchronous DRAM
\item [SRAM] static random access memory
\item [TOS] top of stack
\item [UART] universal asynchronous receiver/transmitter
\item [VHDL] VHSIC hardware description language
\item [VHSIC] very high speed integrated circuit
%\item [WCET] Worst-Case Execution Time
\end{description}



\bibliographystyle{plain}
\bibliography{msbib, chip-design}

\phantomsection
\hypertarget{index}{}
\printindex

\end{document}